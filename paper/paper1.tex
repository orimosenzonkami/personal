%
% File accl-ijcnlp2020.tex
%
%% Based on the style files for ACL 2020, which were
%% Based on the style files for ACL 2018, NAACL 2018/19, which were
%% Based on the style files for ACL-2015, with some improvements
%%  taken from the NAACL-2016 style
%% Based on the style files for ACL-2014, which were, in turn,
%% based on ACL-2013, ACL-2012, ACL-2011, ACL-2010, ACL-IJCNLP-2009,
%% EACL-2009, IJCNLP-2008...
%% Based on the style files for EACL 2006 by 
%%e.agirre@ehu.es or Sergi.Balari@uab.es
%% and that of ACL 08 by Joakim Nivre and Noah Smith

\documentclass[11pt,a4paper]{article}
\usepackage[hyperref]{AACL-IJCNLP2020}
\usepackage{times}
\usepackage{latexsym}
\renewcommand{\UrlFont}{\ttfamily\small}

% This is not strictly necessary, and may be commented out,
% but it will improve the layout of the manuscript,
% and will typically save some space.
\usepackage{microtype}

%\aclfinalcopy % Uncomment this line for the final submission
%\def\aclpaperid{***} %  Enter the aacl-ijcnlp Paper ID here

%\setlength\titlebox{5cm}
% You can expand the titlebox if you need extra space
% to show all the authors. Please do not make the titlebox
% smaller than 5cm (the original size); we will check this
% in the camera-ready version and ask you to change it back.

\newcommand\BibTeX{B\textsc{ib}\TeX}

\title{Instructions for AACL-IJCNLP 2020 Proceedings}

\author{First Author \\
  Affiliation / Address line 1 \\
  Affiliation / Address line 2 \\
  Affiliation / Address line 3 \\
  \texttt{email@domain} \\\And
  Second Author \\
  Affiliation / Address line 1 \\
  Affiliation / Address line 2 \\
  Affiliation / Address line 3 \\
  \texttt{email@domain} \\}

\date{}

\begin{document}
\maketitle
\begin{abstract}
\end{abstract}

\section{Credits}



\section{Introduction}


\section{Electronically-available resources}

AACL provides this description and accompanying style files at
\begin{quote}
\url{http://aacl2020.org//downloads/aacl-ijcnlp2020-templates.zip}
\end{quote}
We strongly recommend the use of these style files, which have been appropriately tailored for the AACL-IJCNLP 2020 proceedings.

\paragraph{\LaTeX-specific details:}
The templates include the \LaTeX2e{} source (\texttt{\small aacl-ijcnlp2020.tex}),
the \LaTeX2e{} style file used to format it (\texttt{\small aacl-ijcnlp2020.sty}),
an ACL bibliography style (\texttt{\small acl\_natbib.bst}),
an example bibliography (\texttt{\small aacl-ijcnlp2020.bib}),
and the bibliography for the ACL Anthology (\texttt{\small anthology.bib}).


\section{Length of Submission}
\label{sec:length}



\section{Anonymity}

\paragraph{\LaTeX-specific details:}


\section{Multiple Submission Policy}


\section{Formatting Instructions}


\paragraph{\LaTeX-specific details:}
The style files will generate page numbers when {\small\verb|\aclfinalcopy|} is commented out, and remove them otherwise.


\subsection{File Format}

\label{sect:pdf}


\paragraph{\LaTeX-specific details:}

\subsection{Layout}
\label{ssec:layout}


\begin{itemize}
\item Left and right margins: 2.5 cm
\item Top margin: 2.5 cm
\item Bottom margin: 2.5 cm
\item Column width: 7.7 cm
\item Column height: 24.7 cm
\item Gap between columns: 0.6 cm
\end{itemize}

\subsection{Fonts}


\begin{table}
\centering
\begin{tabular}{lrl}
\hline \textbf{Type of Text} & \textbf{Font Size} & \textbf{Style} \\ \hline
paper title & 15 pt & bold \\
author names & 12 pt & bold \\
author affiliation & 12 pt & \\
the word ``Abstract'' & 12 pt & bold \\
section titles & 12 pt & bold \\
subsection titles & 11 pt & bold \\
document text & 11 pt  &\\
captions & 10 pt & \\
abstract text & 10 pt & \\
bibliography & 10 pt & \\
footnotes & 9 pt & \\
\hline
\end{tabular}
\caption{\label{font-table} Font guide. }
\end{table}

\paragraph{\LaTeX-specific details:}
To use Times Roman in \LaTeX2e{}, put the following in the preamble:
\begin{quote}
\small
\begin{verbatim}
\usepackage{times}
\usepackage{latexsym}
\end{verbatim}
\end{quote}


\subsection{Ruler}
The camera ready copy should not contain a ruler.

\paragraph{Reviewers:}
note that the ruler measurements may not align well with lines in the paper -- this turns out to be

\paragraph{\LaTeX-specific details:}

\subsection{Title and Authors}
\label{ssec:title-authors}



\subsection{Abstract}

\subsection{Text}


\subsection{Sections}

\subsection{Footnotes}

\subsection{Graphics}

Place figures, tables, and photographs in the paper near where they are first discussed, rather than at the end, if possible.
Wide illustrations may run across both columns.
Color is allowed, but adhere to Section~\ref{ssec:accessibility}'s guidelines on accessibility.

\paragraph{Captions:}
Provide a caption for every illustration; number each one sequentially in the form:
``Figure 1. Caption of the Figure.''
``Table 1. Caption of the Table.''
Type the captions of the figures and tables below the body, using 10 point text.
Captions should be placed below illustrations.
Captions that are one line are centered (see Table~\ref{font-table}).
Captions longer than one line are left-aligned (see Table~\ref{tab:accents}).

\begin{table}
\centering
\begin{tabular}{lc}
\hline
\textbf{Command} & \textbf{Output}\\
\hline
\verb|{\"a}| & {\"a} \\
\verb|{\^e}| & {\^e} \\
\verb|{\`i}| & {\`i} \\ 
\verb|{\.I}| & {\.I} \\ 
\verb|{\o}| & {\o} \\
\verb|{\'u}| & {\'u}  \\ 
\verb|{\aa}| & {\aa}  \\\hline
\end{tabular}
\begin{tabular}{lc}
\hline
\textbf{Command} & \textbf{Output}\\
\hline
\verb|{\c c}| & {\c c} \\ 
\verb|{\u g}| & {\u g} \\ 
\verb|{\l}| & {\l} \\ 
\verb|{\~n}| & {\~n} \\ 
\verb|{\H o}| & {\H o} \\ 
\verb|{\v r}| & {\v r} \\ 
\verb|{\ss}| & {\ss} \\
\hline
\end{tabular}
\caption{Example commands for accented characters, to be used in, \emph{e.g.}, \BibTeX\ names.}\label{tab:accents}
\end{table}

\paragraph{\LaTeX-specific details:}
The style files are compatible with the caption and subcaption packages; do not add optional arguments.
\textbf{Do not override the default caption sizes.}


\subsection{Hyperlinks}
Within-document and external hyperlinks are indicated with Dark Blue text, Color Hex \#000099.

\subsection{Citations}
Citations within the text appear in parentheses as~\citep{Gusfield:97} or, if the author's name appears in the text itself, as \citet{Gusfield:97}.
Append lowercase letters to the year in cases of ambiguities.  
Treat double authors as in~\citep{Aho:72}, but write as in~\citep{Chandra:81} when more than two authors are involved. Collapse multiple citations as in~\citep{Gusfield:97,Aho:72}. 

Refrain from using full citations as sentence constituents.
Instead of
\begin{quote}
  ``\citep{Gusfield:97} showed that ...''
\end{quote}
write
\begin{quote}
``\citet{Gusfield:97} showed that ...''
\end{quote}

\begin{table*}
\centering
\begin{tabular}{lll}
\hline
\textbf{Output} & \textbf{natbib command} & \textbf{Old ACL-style command}\\
\hline
\citep{Gusfield:97} & \small\verb|\citep| & \small\verb|\cite| \\
\citealp{Gusfield:97} & \small\verb|\citealp| & no equivalent \\
\citet{Gusfield:97} & \small\verb|\citet| & \small\verb|\newcite| \\
\citeyearpar{Gusfield:97} & \small\verb|\citeyearpar| & \small\verb|\shortcite| \\
\hline
\end{tabular}
\caption{\label{citation-guide}
Citation commands supported by the style file.
The style is based on the natbib package and supports all natbib citation commands.
It also supports commands defined in previous ACL style files for compatibility.
}
\end{table*}

\paragraph{\LaTeX-specific details:}
Table~\ref{citation-guide} shows the syntax supported by the style files.
We encourage you to use the natbib styles.
You can use the command {\small\verb|\citet|} (cite in text) to get ``author (year)'' citations as in \citet{Gusfield:97}.
You can use the command {\small\verb|\citep|} (cite in parentheses) to get ``(author, year)'' citations as in \citep{Gusfield:97}.
You can use the command {\small\verb|\citealp|} (alternative cite without  parentheses) to get ``author year'' citations (which is useful for  using citations within parentheses, as in \citealp{Gusfield:97}).


\subsection{References}
Gather the full set of references together under the heading \textbf{References}; place the section before any Appendices. 
Arrange the references alphabetically by first author, rather than by order of occurrence in the text.

Provide as complete a citation as possible, using a consistent format, such as the one for \emph{Computational Linguistics\/} or the one in the  \emph{Publication Manual of the American 
Psychological Association\/}~\citep{APA:83}.
Use full names for authors, not just initials.

Submissions should accurately reference prior and related work, including code and data.
If a piece of prior work appeared in multiple venues, the version that appeared in a refereed, archival venue should be referenced.
If multiple versions of a piece of prior work exist, the one used by the authors should be referenced.
Authors should not rely on automated citation indices to provide accurate references for prior and related work.

The following text cites various types of articles so that the references section of the present document will include them.
\begin{itemize}
\item Example article in journal: \citep{Ando2005}.
\item Example article in proceedings, with location: \citep{borschinger-johnson-2011-particle}.
\item Example article in proceedings, without location: \citep{andrew2007scalable}.
\item Example arxiv paper: \citep{rasooli-tetrault-2015}. 
\end{itemize}


\paragraph{\LaTeX-specific details:}
The \LaTeX{} and Bib\TeX{} style files provided roughly follow the American Psychological Association format.
If your own bib file is named \texttt{\small aacl-ijcnlp2020.bib}, then placing the following before any appendices in your \LaTeX{}  file will generate the references section for you:
\begin{quote}\small
\verb|\bibliographystyle{acl_natbib}|\\
\verb|\bibliography{aacl-ijcnlp2020}|
\end{quote}

You can obtain the complete ACL Anthology as a Bib\TeX\ file from \url{https://aclweb.org/anthology/anthology.bib.gz}.
To include both the anthology and your own bib file, use the following instead of the above.
\begin{quote}\small
\verb|\bibliographystyle{acl_natbib}|\\
\verb|\bibliography{anthology,aacl-ijcnlp2020}|
\end{quote}


\subsection{Digital Object Identifiers}
As part of our work to make ACL materials more widely used and cited outside of our discipline, ACL has registered as a CrossRef member, as a registrant of Digital Object Identifiers (DOIs), the standard for registering permanent URNs for referencing scholarly materials.

All camera-ready references are required to contain the appropriate DOIs (or as a second resort, the hyperlinked ACL Anthology Identifier) to all cited works.
Appropriate records should be found for most materials in the current ACL Anthology at \url{http://aclanthology.info/}.
As examples, we cite \citep{goodman-etal-2016-noise} to show you how papers with a DOI will appear in the bibliography.
We cite \citep{harper-2014-learning} to show how papers without a DOI but with an ACL Anthology Identifier will appear in the bibliography.

\paragraph{\LaTeX-specific details:}
Please ensure that you use Bib\TeX\ records that contain DOI or URLs for any of the ACL materials that you reference.
If the Bib\TeX{} file contains DOI fields, the paper title in the references section will appear as a hyperlink to the DOI, using the hyperref \LaTeX{} package.


\subsection{Appendices}
Appendices, if any, directly follow the text and the
references (but only in the camera-ready; see Appendix~\ref{sec:appendix}).
Letter them in sequence and provide an informative title:
\textbf{Appendix A. Title of Appendix}.

\section{Accessibility}
\label{ssec:accessibility}

In an effort to accommodate people who are color-blind (as well as those printing to paper), grayscale readability is strongly encouraged.
Color is not forbidden, but authors should ensure that tables and figures do not rely solely on color to convey critical distinctions.
A simple criterion:
All curves and points in your figures should be clearly distinguishable without color.

\section{Translation of non-English Terms}

It is also advised to supplement non-English characters and terms with appropriate transliterations and/or translations since not all readers understand all such characters and terms.
Inline transliteration or translation can be represented in the order of:
\begin{center}
\begin{tabular}{c}
original-form \\
transliteration \\
``translation''
\end{tabular}
\end{center}

\section{\LaTeX{} Compilation Issues}
You may encounter the following error during compilation: 
\begin{quote}
{\small\verb|\pdfendlink|} ended up in different nesting level than {\small\verb|\pdfstartlink|}.
\end{quote}
This happens when \texttt{\small pdflatex} is used and a citation splits across a page boundary.
To fix this, the style file contains a patch consisting of two lines:
(1) {\small\verb|\RequirePackage{etoolbox}|} (line 455 in \texttt{\small aacl-ijcnlp2020.sty}), and
(2) A long line below (line 456 in \texttt{\small aacl-ijcnlp2020.sty}).

If you still encounter compilation issues even with the patch enabled, disable the patch by commenting the two lines, and then disable the \texttt{\small hyperref} package by loading the style file with the \texttt{\small nohyperref} option:

\noindent
{\small\verb|\usepackage[nohyperref]{aacl-ijcnlp2020}|}

\noindent
Then recompile, find the problematic citation, and rewrite the sentence containing the citation. (See, {\em e.g.}, \url{http://tug.org/errors.html})

\section*{Acknowledgments}

The acknowledgments should go immediately before the references. Do not number the acknowledgments section.
Do not include this section when submitting your paper for review.

\bibliography{anthology,aacl-ijcnlp2020}
\bibliographystyle{acl_natbib}

\appendix

\section{Appendices}
\label{sec:appendix}
Appendices are material that can be read, and include lemmas, formulas, proofs, and tables that are not critical to the reading and understanding of the paper. 
Appendices should be \textbf{uploaded as supplementary material} when submitting the paper for review.
Upon acceptance, the appendices come after the references, as shown here.

\paragraph{\LaTeX-specific details:}
Use {\small\verb|\appendix|} before any appendix section to switch the section numbering over to letters.


\section{Supplemental Material}
\label{sec:supplemental}
Submissions may include non-readable supplementary material used in the work and described in the paper.
Any accompanying software and/or data should include licenses and documentation of research review as appropriate.
Supplementary material may report preprocessing decisions, model parameters, and other details necessary for the replication of the experiments reported in the paper.
Seemingly small preprocessing decisions can sometimes make a large difference in performance, so it is crucial to record such decisions to precisely characterize state-of-the-art methods. 

Nonetheless, supplementary material should be supplementary (rather than central) to the paper.
\textbf{Submissions that misuse the supplementary material may be rejected without review.}
Supplementary material may include explanations or details of proofs or derivations that do not fit into the paper, lists of
features or feature templates, sample inputs and outputs for a system, pseudo-code or source code, and data.
(Source code and data should be separate uploads, rather than part of the paper).

The paper should not rely on the supplementary material: while the paper may refer to and cite the supplementary material and the supplementary material will be available to the reviewers, they will not be asked to review the supplementary material.

\end{document}
